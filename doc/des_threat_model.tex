\section{Threat model}
The focus of this project is to design and implement a separation kernel, which
guarantees strong separation of subjects and can thus serve as a basis for a
component-based system. This section describes the capabilities of an attacker
and enumerates the issues that are considered out of scope.

\begin{itemize}
	\item All code that is not part of the TCB can potentially be subverted.
		This specifically includes \emph{all} untrusted subjects. Thus an
		attacker can completely control all untrusted subjects.
\end{itemize}

The following issues, while very important in the context of constructing a
high assurance system, are considered out of scope:

\begin{description}
	\item[System initialization] The kernel starts executing after the
		bootloader hands over execution. It is assumed that the system is setup
		and initialized properly. How the system is securely bootstrapped (e.g.
		using a trusted boot process) and initialized and how the integrity of
		the kernel is assured are not considered.
	\item[Hardware] It is assumed, that hardware such as the CPU, memory
		management unit and other devices are working correctly and according to
		their specification. Problems due to buggy or even malicious hardware
		are considered out of scope.
	\item[Physical attacks] Issues predicated on an attacker having physical
		access to the system are not considered.
	\item[Firmware] A modern PC contains firmware and many embedded controllers,
		that are only partly (if at all) controllable by an operating system
		kernel. This includes technologies such as Intel AMT/ME, System
		Management Mode (SMM) and the system BIOS, which have access to
		sensitive system resources.
	\item[Policy validity] The separation kernel provides the mechanisms to
		enforce a given, user-defined policy. The focus is on the correct
		implementation of a provided system configuration. The user is in charge
		of assuring the correctness and consistency of the overall system
		policy.
	\item[Communication] The separation kernel must provide a
		mechanism to establish directed communication channels between subjects.
		It is however not the duty of the kernel to provide a inter-subject
		communication abstraction such as message passing or a remote procedure
		call (RPC) interface.
	\item[Recovery] How a system can be restored to a secure state after a
		compromise is out of scope.
\end{description}

These points must be considered and addressed when building a high assurance
system based on the Muen separation kernel.
