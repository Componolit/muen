\section{Intel x86 Architecture}
Even though a more detailed familiarity with the subject is required for a full
understanding of the design and implementation of the Muen kernel, this section
only tries to give an short tour of the Intel x86 architecture. The interested
reader is directed to the Intel Software Developer's Manuals \cite{IntelSDM},
which contain a complete and in-depth description of the x86 architecture.

The basic components of a modern Intel x86 computer are depicted in figure
\ref{fig:intel-architecture}.

\begin{figure}
	\begin{tikzpicture}
	\node[apribox, minimum width=3cm, minimum height=1cm] (cpu) {CPU};
	\node[bluebox] (gfx) [left=of cpu] {PCIe/Graphics Adapter};
	\node[bluebox] (ram) [right=of cpu] {RAM};

	\node[commonbox, minimum width=3cm, minimum height=1cm] (pch) [below=1cm of cpu] {PCH};
	\node[greenbox] (network) [left=of pch] {Network};
	\node[greenbox] (usb) [below left=of pch] {USB};
	\node[greenbox] (pci) [below=of pch] {PCI Express};
	\node[greenbox] (bios) [below right=of pch] {BIOS};
	\node[greenbox] (legacy) [right=of pch] {Legacy};

	\draw[thick] (cpu) to node[auto] {DMI} (pch);
	\draw[thick] (cpu) -- (gfx);
	\draw[thick] (cpu) -- (ram);
	\draw[thick] (pch) -- (network);
	\draw[thick] (pch) -- (usb);
	\draw[thick] (pch) -- (pci);
	\draw[thick] (pch) -- (bios);
	\draw[thick] (pch) -- (legacy);
\end{tikzpicture}

	\caption{Intel architecture}
	\label{fig:intel-architecture}
\end{figure}

\subsection{Processor}
The main processor of the system is called the central processing unit (CPU). On
multi-processor systems a CPU can have multiple cores, which are termed
\emph{logical CPUs}. Each logical CPU fetches and executes instructions, which
change the state of the processor and the system.

\subsubsection{Execution environment}
A program executing on an processor is given access to resources to store
information, such as code, data or state information. Figure
\ref{fig:64bit-exec-env} illustrates the basic execution environment of a modern
64-bit processor.

\begin{figure}
	\caption{64-bit execution environment}
	\label{fig:64bit-exec-env}
\end{figure}

The various types of resources are described in the following list:

\begin{description}
	\item[Address space] is used by a program to access system memory.
	\item[Stack] is located in memory and facilitates subprogram calls and
		parameter passing in conjunction with stack management resources.
	\item[Execution registers] constitute the core of the execution environment.
		It consists of sixteen general-purpose, six segment and one flag
		register as well as the instruction pointer.
	\item[Control registers] define the current processor operating mode and
		other properties of the execution environment.
	\item[Descriptor table registers] are used to store the location of memory
		management and interrupt handling data structures.
	\item[Debug registers] allow a program to control and use the debugging
		functionality of a processor.
	\item[x87 FPU registers] offer support for floating point operations.
	\item[MMX registers] offer support for single-instruction, multiple-data
		(SIMD) operations for integer numbers.
	\item[XMM registers] offer support for SIMD operations on floating-point
		numbers.
	\item[Model-specific registers (MSRs)] provide control of various hardware
		and software-related features. Their number and functionality differs
		depending on the given processor implementation.
	\item[I/O ports] allow transfer of data to and from Input/Output ports.
\end{description}

All instructions provided by a processor are called an instruction set. The
complete Intel instruction set architecture (ISA)\index{ISA} is specified in
\cite{IntelSDM}, volumes 2a-c.

\subsubsection{Caches}
The processor has numerous internal caches and buffers of varying sizes and
properties. Their main purpose is to increase processor performance by
optimizing accesses, e.g. reading data from system memory. The most important
caches are outlined in the following list:

\begin{itemize}
	\item Level 1 instruction cache
	\item Level 1 data cache
	\item Level 2 \& 3 unified caches
	\item Translation lookaside buffers (TLB)
	\item Store buffer
	\item Write Combining buffer
\end{itemize}

Since caches are shared and because they can only be controlled to a limited
degree, their state can be changed and observed by different parts of a system.
Thus some of the caches can be used as side-/covert-channels and pose a
challenge to component isolation.

\subsubsection{Protected mode}
A modern CPU provides several operating modes of which one is called
\emph{protected mode}. In this mode the processor provides different privilege
levels also called rings. The rings are numbered from 0 to 3, with ring 0 having
the most privileges. Due to common operating systems using only rings 0 and 3
and disregarding other privilege levels, the 64-bit extension to the Intel
architecture dropped support for rings 1 and 2.

Privileged instructions such as switching memory management data structures are
only executable in ring 0 also called supervisor mode. Operating systems such as
Linux, usually execute user applications in the unprivileged ring 3, called user
mode.

\subsubsection{Memory management}
Current processors support management of physical memory through means of
segmentation and paging. Logical addresses are translated to linear addresses
using the segmentation mechanism. In a second step, linear addresses are
translated to physical addresses using paging.

Memory management is done in hardware by the memory management unit (MMU)\index{MMU}.
An operating system must set up certain data structures (descriptor tables and
page tables) to instruct the MMU how logic and linear memory addresses map to
physical memory. The MMU allows for partitioning and separation of system
memory.

A processor is connected to the memory controller hub via the system bus, which
is also called front side bus (FSB)\index{FSB}.

\subsection{Memory Controller Hub}
The memory controller hub (MCH)\index{MCH}, also called \emph{Northbridge},
connects the CPU to random access memory (RAM), the graphics card and
southbridge (presented in the next section) of the system. The processor
accesses these resources by sending memory requests via the system bus. The MCH
decodes the requests and forwards them to RAM or the Southbridge, depending on
the physical address. If for example a request contains an address of a
memory-mapped device, the chipset of the MCH forwards the request to the
Southbridge.

\subsection{Input/Output Controller Hub}
The input/ouput controller hub (ICH), also called \emph{Southbridge}, provides
connections to most devices and platform peripherals, such as keyboards, USB
connections and other buses such as PCI Express, USB etc. Newer models contain
an I/O memory management unit (IOMMU), which provides address translation
functionality similar to the MMU, but for accesses to system memory initiated by
I/O devices. This is necessary since active components, such as PCI devices, can
perform direct memory access (DMA)\index{DMA}, which bypasses the processor's
MMU memory protection. Like the MMU, an operating system must initialize data
structures and setup the IOMMU for proper separation of physical memory
accessible by devices.

\subsection{Platform Controller Hub}
More recently, Intel has rearranged the presented hub architecture and has
integrated much of the functionality provided by the Northbridge into the CPU
\cite{IntelQPI}. All remaining functions of the Northbridge and Southbridge have
been combined and incorporated into a new component called the platform
controller hub (PCH) \index{PCH}. The is new architecture avoids the bottleneck
presented by the system bus connection between the CPU and the Northbridge.
