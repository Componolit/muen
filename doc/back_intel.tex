\section{Intel x86 Architecture}
Even though a more detailed familiarity with the subject is required for a full
understanding of the design and implementation of the Muen kernel, this section
tries to give a short tour of the Intel x86 architecture. The interested reader
is directed to the Intel Software Developer's Manuals (SDM\index{SDM})
\cite{IntelSDM}, which contain a complete and in-depth description of the x86
architecture.

The basic components of a modern Intel x86 computer are depicted in figure
\ref{fig:intel-architecture}.

\begin{figure}[h]
	\centering
	\begin{tikzpicture}
	\node[apribox, minimum width=3cm, minimum height=1cm] (cpu) {CPU};
	\node[bluebox] (gfx) [left=of cpu] {PCIe/Graphics Adapter};
	\node[bluebox] (ram) [right=of cpu] {RAM};

	\node[commonbox, minimum width=3cm, minimum height=1cm] (pch) [below=1cm of cpu] {PCH};
	\node[greenbox] (network) [left=of pch] {Network};
	\node[greenbox] (usb) [below left=of pch] {USB};
	\node[greenbox] (pci) [below=of pch] {PCI Express};
	\node[greenbox] (bios) [below right=of pch] {BIOS};
	\node[greenbox] (legacy) [right=of pch] {Legacy};

	\draw[thick] (cpu) to node[auto] {DMI} (pch);
	\draw[thick] (cpu) -- (gfx);
	\draw[thick] (cpu) -- (ram);
	\draw[thick] (pch) -- (network);
	\draw[thick] (pch) -- (usb);
	\draw[thick] (pch) -- (pci);
	\draw[thick] (pch) -- (bios);
	\draw[thick] (pch) -- (legacy);
\end{tikzpicture}

	\caption{Intel architecture}
	\label{fig:intel-architecture}
\end{figure}

\subsection{Processor}
The main processor of the system is called the central processing unit
(CPU\index{CPU}). Multi-processor systems (MP\index{MP}) have multiple CPUs
which in term can have multiple cores. Each core in each CPU is a so called
\emph{logical CPU} which executes instructions. These instructions change the
state of the logical CPU and the system as a whole.

\subsubsection{Execution environment}\label{subsubsec:exec-env}
A program executing on an processor is given access to resources to store
information, such as code, data or state information. Resources provided by a
modern 64-bit processor constituting the basic execution environment are
described in the following list:

\begin{description}
	\item[Address space] is used by a program to access system memory.
	\item[Stack] is located in memory and facilitates subprogram calls and
		parameter passing in conjunction with stack management resources of the
		processor.
	\item[Execution registers] constitute the core of the execution environment.
		It consists of sixteen general-purpose, six segment and one flag
		register as well as the instruction pointer.
	\item[Control registers] define the current processor operating mode and
		other properties of the execution environment.
	\item[Descriptor table registers] are used to store the location of memory
		management and interrupt handling data structures.
	\item[Debug registers] allow a program to control and use the debugging
		functionality of a processor.
	\item[x87 FPU registers] offer support for floating point operations.
	\item[MMX registers] offer support for single-instruction, multiple-data
		(SIMD) operations for integer numbers.
	\item[XMM registers] offer support for SIMD operations on floating-point
		numbers.
	\item[Model-specific registers (MSRs)] provide control of various hardware
		and software-related features. Their number and functionality varies
		depending on the given processor implementation.
	\item[I/O ports] allow transfer of data to and from Input/Output ports.
\end{description}

Further details about the Intel processor execution environment can be found in
\cite{IntelSDM}, volume 1, section 3.2.1. All instructions provided by a
processor are called an instruction set. The complete Intel instruction set
architecture (ISA)\index{ISA} is specified in \cite{IntelSDM}, volumes 2A-C.

\subsubsection{Caches}
The processor has numerous internal caches and buffers of varying sizes and
properties. Their main purpose is to increase processor performance by
optimizing accesses, e.g. reading data from system memory. The most important
caches are outlined in the following list:

\begin{itemize}
	\item Level 1 instruction cache
	\item Level 1 data cache
	\item Level 2 \& 3 unified caches
	\item Translation lookaside buffers (TLB)
	\item Store buffer
	\item Write Combining buffer
\end{itemize}

Since caches are shared and because they can only be controlled to a limited
degree, their state can be changed and observed by different parts of a system.
Thus some of the caches can be used as side-/covert-channels and pose a
challenge to effective component isolation.

\subsubsection{Protected mode}
A modern CPU provides several operating modes of which one is called
\emph{protected mode}. In this mode the processor provides different privilege
levels also called rings. The rings are numbered from 0 to 3, with ring 0 having
the most privileges. Due to common operating systems using only rings 0 and 3
while disregarding other privilege levels, the 64-bit extension to the Intel
architecture dropped support for rings 1 and 2.

Privileged instructions such as switching memory management data structures are
only executable in ring 0, also called supervisor mode. Operating systems such
as Linux, usually execute user applications in the unprivileged ring 3, called
user mode.

\subsubsection{Memory management}
Current processors support management of physical memory through means of
segmentation and paging. Logical addresses are translated to linear addresses
using the segmentation mechanism. In a second step, linear addresses are
translated to physical addresses using paging.

Memory management is done in hardware by the memory management unit
(MMU)\index{MMU}.  An operating system must set up certain data structures
(descriptor tables and page tables) to instruct the MMU how logic and linear
memory addresses map to physical memory. The MMU allows for partitioning and
separation of system memory.

A processor is connected to the memory controller hub via the system bus, which
is also called front side bus (FSB)\index{FSB}.

\subsubsection{Exceptions and interrupts}
TODO: IDT handling etc

\subsection{Memory Controller Hub}
The memory controller hub (MCH)\index{MCH}, also called
\emph{Northbridge}\index{northbridge}, connects the CPU to random access memory
(RAM), the graphics card and southbridge (presented in the next section) of the
system. The processor accesses these resources by sending memory requests via
the system bus. The MCH decodes the requests and forwards them to RAM or the
Southbridge, depending on the physical address. If for example a request
contains an address of a memory-mapped device, the chipset of the MCH forwards
the request to the Southbridge.

\subsection{Input/Output Controller Hub}\label{subsec:ich}
The input/ouput controller hub (ICH\index{ICH}), also called
\emph{Southbridge}\index{southbridge}, provides connections to most devices and
platform peripherals, such as keyboards, USB connections and other buses (PCI
Express etc). Newer models contain an I/O memory management unit
(IOMMU\index{IOMMU}), which provides address translation functionality similar
to the MMU, but for accesses to system memory initiated by I/O devices.  This is
necessary since PCI devices perform direct memory access (DMA)\index{DMA}, which
bypasses the processor's MMU memory protection.  Like the MMU, an operating
system must initialize data structures and setup the IOMMU for proper separation
of physical memory accessible by devices.

\subsection{Platform Controller Hub}
More recently, Intel has rearranged the presented hub architecture and has
integrated much of the functionality provided by the Northbridge into the CPU
\cite{IntelQPI}. All remaining functions of the Northbridge and Southbridge have
been combined and incorporated into a new component called the platform
controller hub (PCH)\index{PCH}. This new architecture avoids the bottleneck
presented by the system bus connection between the CPU and the Northbridge.

\subsection{Programmable Interrupt Controller}\label{subsec:apic}
A Programmable Interrupt Controller (PIC\index{PIC}) is used to connect several
input sources to a CPU. Hardware devices raise interrupts to inform a CPU that
some event occurred which must be handled. For example a network card notifies
the CPU that it received data which must be processed. One of the best known PIC
is the Intel 8259A which was part of the original PC\footnote{Personal
computer}\index{PC} introduced in 1981.

Early PC/XT ISA\index{ISA} systems used one 8259 controller, allowing only eight
interrupt input lines. By cascading multiple controllers, more lines can be made
available but nevertheless, a more flexible approach was needed with the advent
of multiprocessor (MP\index{MP}) systems containing multiple processor cores.
Implementing efficient interrupt routing on an MP system using PICs was an
issue.

Intel presented the Advanced Programmable Interrupt Controller
(APIC\index{APIC}) concept with the introduction of the Pentium processor. It
was one of several attempts to solve the interrupt routing efficiency problem.
The Intel APIC system is composed of two components: a local APIC
(LAPIC\index{LAPIC}) in each CPU of the system and the I/O APIC, see figure
\ref{fig:apic} for a schematic view.

\begin{figure}[h]
	\centering
	\begin{tikzpicture}
	\node (du8) {};
	\node[bluebox, minimum width=2cm, minimum height=1.4cm, label={[yshift=-18pt]CPU0}, left=5mm of du8] (cp0) {};
	\node[graybox] (la0) at (cp0.south) [above=5pt] {LAPIC};
	\node[bluebox, minimum width=2cm, minimum height=1.4cm, label={[yshift=-18pt]CPU1}, right=5mm of du8] (cp1) {};
	\node[graybox] (la1) at (cp1.south) [above=5pt] {LAPIC};

	\node[align=center, left=8mm of la0]  (li0) {Local\\Interrupts};
	\node[align=center, right=8mm of la1] (li1) {Local\\Interrupts};

	\node[inner sep=0, below=6mm of cp0] (du0) {};
	\node[inner sep=0, below=6mm of cp1] (du1) {};

	\draw[arrow] (li0) -- (la0);
	\draw[arrow] (li1) -- (la1);
	\draw[thick, <->] (cp0) -- (du0);
	\draw[thick, <->] (cp1) -- (du1);
	\draw[very thick] (du0) to node[auto, name=sys] {System Bus} (du1);

	\node[apribox, below=6mm of sys]  (ioa) {I/O APIC};
	\node[align=center, right=of ioa] (ext) {External\\Interrups};

	\draw[thick, <->] (ioa) -- (sys);
	\draw[arrow] (ext) -- (ioa);
\end{tikzpicture}

	\caption{Local APICs and I/O APIC}
	\label{fig:apic}
\end{figure}

A LAPIC receives interrupts from internal sources (such as the timer interrupt)
and from the I/O APIC or other external interrupt controllers. It sends these
interrupts to the processor core for handling. In MP systems the LAPIC also
sends and receives inter-processor interrupt messages (IPI\index{IPI}) from and
to other logical processors on the system bus.

The I/O APIC is used to receive external interrupts from the system and its
associated devices and relay them to the local APICs. The LAPICs are connected
to the I/O APIC via system bus. The routing of interrupts to the appropriate
LAPICs is configured using a redirection table which must be setup by the
operating system. For more information about APIC and I/O APIC see chapter 10 in
\cite{IntelSDM}.
