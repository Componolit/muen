\section{Virtualization}
Virtualization is an established architectural concept in computer science and
has been in use since decades. Operating systems for example provide virtual
address spaces to application processes, giving them the illusion of unlimited,
continuous memory. An application does not need to take care of complex memory
management tasks and the operating system is able to optimize the usage of
physical memory. Another common example is the virtualization of devices such as
virtual CD-ROM drives directly using a file-based backend for data I/O
\cite{CryptoCloud}.

Hardware or platform virtualization is the process of creating virtual computer
hardware that acts like real hardware. The virtualization is performed on the
host machine by special software, called a hypervisor\index{hypervisor} or
virtual machine monitor (VMM\index{VMM}). These two terms are synonyms, for
consistency we will use VMM throughout this document. VMMs are classified into
two types see figure \ref{fig:vmm-classification}.

\begin{figure}
	\centering
	\begin{subfigure}[b]{0.24\textwidth}
		\centering
		\begin{tikzpicture}[text width=2cm, minimum height=0.5cm]
	\node (haw) [apribox]                     {Hardware};
	\node (vmm) [bluebox, above=0.1cm of haw] {VMM};
	\node (vim) [greenbox, above=of vmm]      {Virtual Machine};

	\draw[arrow] (vmm) to (vim);
\end{tikzpicture}

		\caption{\emph{Type I, native or bare metal VMM.} Runs directly on the
		hardware in the most privileged processor mode.}
	\end{subfigure}
	\qquad
	\begin{subfigure}[b]{0.24\textwidth}
		\centering
		\begin{tikzpicture}[text width=2cm, minimum height=0.5cm]
	\node (haw) [apribox]                     {Hardware};
	\node (hos) [apribox, above=0.1cm of haw] {Hosting OS};
	\node (vmm) [bluebox, above=0.1cm of hos] {VMM};
	\node (vim) [greenbox, above=of vmm]      {Virtual Machine};

	\draw[arrow] (vmm) to (vim);
\end{tikzpicture}

		\caption{\emph{Type II or hosted VMM.} The VMM runs on top of a
		conventional operating system and uses OS services.}
	\end{subfigure}
	\caption{VMM classification}
	\label{fig:vmm-classification}
\end{figure}

Software that runs in a virtualized environment is called guest software. The
words host and guest are used to distinguish the software that runs on the
physical machine from the software that runs on the virtual machine
\cite{wiki:virtualization}.

By using virtualization and scheduling techniques, a VMM multiplexes the real
hardware of a computer making it possible to run multiple guests in
parallel\footnote{On multicore machines in parallel, on single-core machines
pseudo-parallel} on one machine. This is the most common use-case for
virtualization: The consolidation of operating system instances on one server to
save power and to improve hardware-resource utilization.

A guest program is separated from the real hardware of a machine, direct access
is restricted and can be controlled by the host VMM. Since the VMM has complete
control over the hardware of a system and all guest software, virtualization is
also useful for separation purposes. The VMM can allow certain communication
channels between guest software while restricting others. One guest could have
access to the network card, while others share a page in memory but have no
access to hardware devices.

The principle idea of virtualization is to run guest code in the virtual machine
and intercept privileged operations which access critical resources or system
properties. A guest traps into the VMM if such an instruction is executed and
the VMM acts according to the situation. This method is used to implement a
technique called "trap and emulate". For example if a guest accesses a device
which is emulated by the VMM, access to the virtual device causes a trap into
the VMM. The VMM itself or a special monitor VM then emulates the requested
operations of the guest software by directly modifying the state of the virtual
processor or memory of the guest VM.

A trap from the guest into the VMM is a costly operation. To reduce traps,
mechanisms in modern processors have been introduced to support the VMM in
creating a virtual machine environment. Processor features like Intel VT-x, EPT
and VT-d not only improve the performance of a virtual machine by avoiding traps
but also allow the VMM software code to be much simpler.

