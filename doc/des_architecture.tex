\section{Architecture}
The core mechanism used to separate subjects is Intel's hardware-assisted
virtualization technology VT-x. The kernel executes in VMX root mode, while
subjects run in VMX non-root mode. This shields the kernel from access by
subjects. Figure \ref{fig:architecture-overview} illustrates the basic system
architecture: the Muen kernel executes two isolated subjects that have no
access to any kernel resources. The native subject is a bare bones application
and the second subject is a virtual machine (VM) type subject, e.g. an
operating system.

\begin{figure}[h]
	\centering
	\begin{tikzpicture}
	\node[bluebox, minimum height=1cm, text width=5cm] (mue) {Muen Separation Kernel};
	\node[greenbox, minimum height=2cm, text width=2cm, above=of mue.north west, anchor=south west] (nat) {Native Subject};
	\node[greenbox, minimum height=2cm, text width=2cm, above=of mue.north east, anchor=south east] (vim) {VM Subject};
	\draw[gray] (-2.6,0.5) to (2.6,0.5);
	\draw[gray] (0,0.5) to (0,3);
\end{tikzpicture}

	\caption{Architecture overview}
	\label{fig:architecture-overview}
\end{figure}

Resource management and assignment to subjects is static and done prior to the
execution of the system. All available resources provided by a platform running
the Muen kernel must be specified in a system policy. Similarly, scheduling
plans and all subjects running on the system are part of the policy.

This relieves the kernel from dynamic resource management and avoids the
associated complexity, which greatly simplifies the kernel design and in turn
its implementation.

The policy is described in detail in section \ref{subsec:policy}.

\subsection{Subject execution}
The kernel initializes the subject execution environment and state accoring to
the subject specification. The execution is started from the subject entry point
and continues for a predefined period of time (see \ref{subsec:scheduling}).
Once the time has passed, a trap into the kernel occurs and the current state of
the subject is saved for later resumption.

A subject is constrained to the environment specified by the subject profile and
the resources assigned to it by the policy. If a subject performs an illegal
resource access or performs an operation not allowed by the profile, a trap
occurs and the kernel is invoked. The kernel can then determine the cause for
the transition and handle the condition properly according to policy.

\subsubsection{Subject monitor}
When a subject tries to access resources such as devices that must be emulated,
a system component must perform the necessary actions and change the subject's
system state accordingly. This is to give the subject the impression that it
has unrestricted access to a device while in reality the necessary operations
are effectively emulated by another component.

In a system based on the Muen kernel, this monitoring function can be
implemented by a subject termed \emph{subject monitor}. Such a monitoring
subject can be given access to the state of another subject, effectively
allowing it to change the system state of the monitored subject. A so called
\emph{trap table}, which is part of the subject specification, allows to
specify that execution should be handed over to a subject monitor when a trap
occurs.

Historically this monitoring function has been implemented as part of the
hypervisor of the virtualization system. Since emulation operations can be very
involved and quickly grow in complexity, it is very desirable to extract this
functionality from the separation kernel and thus from the TCB. Since subjects
are confined by the separation kernel the subject monitor concept achieves this
property.

\subsection{Policy}\label{subsec:policy}
To best fulfill the requirements, a system using the Muen kernel uses static
resource assignment and system specification. The main idea is to have a
complete description of the system including all resources such as system
memory, devices and subjects in the form of a policy. The defined resources can
be assigned to subjects. Since the resources and subjects are fixed at
integration time, the policy can be analyzed and validated prior to execution.

The following list encompasses the system policy:
\begin{itemize}
	\item Hardware resources
	\item Kernel specification
	\item Subject specifications
	\item Scheduling
\end{itemize}

\subsubsection{Memory}
All memory resources of a system are static and are explicitly specified in the
system policy. Apart from a few special memory regions, such as the application
processor trampoline (see section \ref{subsec:init}), there are no implicitly
allocated data structures. This allows to determine the exact memory layout of
the final system at integration time.

The kernel and each subject specifies its memory layout. The memory layout
defines which physical memory ranges are accessible, their location in the
virtual address space of the binary and additional attributes. These attributes
define read/write and executable properties of the memory region and the caching
behavior.

Since the layout is controlled by memory management data structures,
each subject has its own distinct set of page tables. To assure that a subject
cannot alter the memory layout, it must not have access to any page tables,
including its own. This is achieved by not mapping memory management structures
into the address space of a subject.

Figure \ref{fig:phys-mem-layout-example} illustrates the physical memory
structure of an example system.

\begin{figure}[h]
	\centering
	\begin{bytefield}{24}
	\bitbox[]{10}{}
	\bitbox[]{16}{$\vdots$}\\[1ex]
	\begin{rightwordgroup}{$\tau0$}
		\memsection{0021 6000}{0021 9fff}{2}{Code and data}\\
		\memsection{0021 4000}{0021 5fff}{2}{I/O bitmap}\\
		\memsection{0021 0000}{0021 3fff}{2}{Page table}
	\end{rightwordgroup}\\
	\memsection{0020 4000}{0020 ffff}{3}{\color{Gray}--  free --}\\
	\begin{rightwordgroup}{Kernel}
		\memsection{0020 0000}{0020 3fff}{2}{Kernel page table}\\
		\memsection{001f f000}{001f ffff}{2}{$\tau0\rightarrow$kernel interface}\\
		\memsection{001f e000}{001e ffff}{2}{Subject state descriptors}\\
		\memsection{0011 c000}{001f dfff}{3}{\color{Gray}-- free --}\\
		\memsection{0010 0000}{0011 bfff}{2}{Kernel code and data}
	\end{rightwordgroup}\\
	\bitbox[]{10}{}
	\bitbox[]{16}{$\vdots$}
\end{bytefield}

	\caption{Physical memory layout example}
	\label{fig:phys-mem-layout-example}
\end{figure}

The physical memory addresses given on the left side are hexadecimal values.
The memory region containing kernel code and data starts at address
\texttt{0x100000} (1 Megabyte (MB)) and has a size of 112 kilobytes (KB). It is
followed by a gap of unallocated memory and then the "subject
descriptors"\footnote{Internal kernel array used to store subject states} and
"$\tau0\rightarrow$ kernel interface" regions which are each 4 KB in size.
Memory management data structures of the kernel start at address
\texttt{0x200000} (2 MB) and expand over 16 KB. The memory layout of subject
$\tau0$ encompasses three regions ranging from \texttt{0x210000} to
\texttt{0x0219fff}.

Since the address space of a subject cannot change, the page tables are static
as well, which means they can be generated according to the relevant information
in the system policy. Whenever a subject is executed on the CPU, the kernel
directs the MMU to use the corresponding paging structures. The hardware memory
management mechanism then enforces the address translations specified by the
page tables, ultimately restricting the subject to the virtual address space
declared by the policy. Figure \ref{fig:virt-mem-layout-example} illustrates the
memory layout of an example subject.

\begin{figure}[h]
	\centering
	\newcommand{\vmemsection}[6]{
	\bytefieldsetup{bitheight=#5\baselineskip} % define the height of the memsection
	\bitbox[]{8}{
		\footnotesize\texttt{0x#2} % print virtual end address
		\\ \vspace{#5\baselineskip} \vspace{-2\baselineskip} \vspace{-#5pt}
		\footnotesize\texttt{0x#1} % print virtual start address
	}
	\bitbox{14}{#6} % print box with caption
	\bitbox[]{8}{
		\footnotesize\texttt{0x#4} % print physical end address
		\\ \vspace{#5\baselineskip} \vspace{-2\baselineskip} \vspace{-#5pt}
		\footnotesize\texttt{0x#3} % print physical start address
	}
}

\begin{bytefield}{24}
	\vmemsection{001f f000}{001f ffff}{0010 0000}{0010 0fff}{2}{Subject interface}\\
	\vmemsection{0021 9000}{0021 9fff}{0000 3000}{0000 3fff}{2}{Stack}\\
	\vmemsection{0021 8000}{0021 8fff}{0000 2000}{0000 2fff}{2}{Data}\\
	\vmemsection{0021 7000}{0021 7fff}{0000 1000}{0000 1fff}{2}{Read-only data}\\
	\vmemsection{0021 6000}{0021 6fff}{0000 0000}{0000 0fff}{2}{Program code}\\
\end{bytefield}

	\caption{Virtual memory layout of example subject}
	\label{fig:virt-mem-layout-example}
\end{figure}

The size and the physical as well as the virtual address of each memory region
is specified in the policy.

\subsubsection{Devices}
A device can be modeled as a collection of system resources such as I/O ports,
system memory for memory-mapped IO (MMIO) or hardware interrupts. A PS/2
keyboard for example is composed of two I/O ports and a hardware interrupt.  In
order for subjects to interact with a device, it must be able to access the I/O
ports and process the interrupts raised by the device. Thus a policy device
specification links a device name with a set of hardware resources. Assignment
to subjects is done via references to devices in the subject specification.
Devices that are not allocated to a subject are not accessible during the
runtime of the system.

Like any other resource, care must be taken if a device is shared by multiple
subjects since it can be used to transfer information. This is true for any
shared resource.

\subsubsection{Processor}
Information about the processor of the system must also be specified in the
policy, since the number of available logical CPUs for example is crucial for
scheduling. This allows to verify the consistency of the scheduling plan with
regards to the processor that the system is executing on.

\subsubsection{Subject specification}
All subjects of a system must be specified as part of the policy. The subject
specification defines the resources and properties of a subject. Each subject is
identified by a unique id and a name. The name is used for subject references in
the policy, e.g. event table entries specify a destination subjects which is
given by name. Most subject data structures are organized in static arrays and
the id can be used as index to get the information associated with a specific
subject.

The policy also defines what execution environment the kernel must provide by
assigning each subject a profile. Initial values for parts of the execution
environment (stack and instruction pointer) must also be provided by the subject
specification. The binary file constituting the executable code of the subject
and its memory location must also be defined.

Another important part is the assignment of resources. This includes the
designation of the executing CPU, references to granted hardware devices and the
memory layout of the virtual address space. Access to I/O ports and
model-specific registers (MSRs) is also part of the subject specification.

There are two tables that specify how certain conditions caused by a subject
are to be handled by the kernel: the trap and event tables.

The \emph{trap table} designates, what destination subject is in charge of
handling a specific trap and what interrupt vector should be injected, if any.
It defines to which target subject execution control is transfered when a trap
occurs. The optional injection of an interrupt allows indicate the trap kind
to the destination subject, so it can distinguish different causes for traps.

Events, which are described in section \ref{subsubsec:events}, are specified in
the \emph{event table}. An event is a trap that is explicitly triggered by a
subject. The event table specifies how the kernel is supposed to handle the
event.

Both tables are part of the subject policy because the information is subject
specifc. This allows to specify different reactions to traps depending on the
subject causing the condition.

\subsection{Inter-subject communication}
The Muen kernel provides two main mechanisms for information flow between
subjects: shared memory and so called events. The first mechanism can be used
to share arbitrary data between subjects while events are a method to pass
notifications between subjects in the form of injected interrupts.

All communication paths, be it shared memory regions or events, are declared as
part of the subject specification in the system policy. This means that all
channels that provide a way for subjects to exchange information are explicit
and cannot change during the runtime of the system. This allows the validation
of inter-subject information flows prior to execution.

\subsubsection{Shared memory}
The main mechanism for subjects to exchange data is to specify a common shared
memory region in the system policy. A memory region is shared, if two or more
subject memory layouts specify a region that map the same physical memory range
into the respective address spaces.

Access rights for memory regions (e.g. write access) are part of the subject
memory layout. This allows to specify a direction of information flow for memory
regions, e.g. by granting write access to the source subject exclusively and
giving the destination subject read access only.

Since memory management is static and exclusively governed by the system policy,
the kernel does not need to make special provisions. The kernel is oblivious to
shared memory regions, it simply installs the page tables generated by the
policy compiler.

The kernel does not provide a higher level abstraction than shared memory.
Subjects must implement their own protocol, such as message passing, based on
shared memory regions and the event mechanism.

\subsubsection{Events}\label{subsubsec:events}
TODO: Explain event type (handover, interrupt), definition of handover.

An event is caused by a subject and leads to the delivery of an interrupt to a
specific destination subject. The subject triggering an event is called the
source subject. This facilitate the implementation of a simple inter-subject
notification mechanism and enables event-driven services.

Events are explicitly triggered by a given source subject and are part of the
subject specification. The subject policy contains a list of event table
entries which state the event number, destination subject and interrupt number
to inject.

When a subject triggers an event, it invokes a hypercall\footnote{A hypercall is
a software trap from a subject to the VMM. It is initiated using the
\texttt{VMCALL} instruction.} passing an event number as parameter to the
kernel. The kernel then consults the subject's event table to determine the
destination subject and vector. If a valid entry is present, an interrupt with
the specified interrupt number is injected into the destination subject. If the
event number is invalid (e.g. not in the range of the subject's event table) the
event is ignored.

\subsection{Exceptions}\label{subsec:design-exceptions}
We distinguish hardware exceptions that occur in VMX non-root mode, while
executing a subject, and in VMX root mode when the kernel is operating.

As the kernel is implemented in the SPARK programming language and the absence
of runtime errors is proven, exceptions during regular operation in VMX
root-mode should not happen. If for some unforseen reason (e.g. non-maskable
interrupt NMI) an exception occurs, it indicates a serious error and the system
is halted.

In the case of an exception being caused by the execution of a subject, the
exception handling depends on the profile of the running subject. If a native
subject raises an exception, a transition to VMX root-mode occurs with the exit
reason indicating the cause. The kernel then consults the subject's trap table
to determine which subject is in charge of handling the condition.

VM subjects are able to perform their own exception handling. Thus if such a
subject raises an exception, it is delivered to the subject's exception handler
via the IDT (see section \ref{subsec:exceptions-interrupts}.

A trap occurs if the subject is somehow not able to handle the exception
properly and a triple fault occurs (e.g. a nested exception occurs in the
exception handler and the subject has not installed a double fault handler).
The trap is then processed by the kernel like any other trap caused by the
subject by using the trap table to schedule the destination subject according to
the policy.

\subsection{Interrupts}
Devices can generate hardware interrupts that must be delivered to the subject
that controls the device according to the policy. A device specification in the
system policy defines which hardware interrupt it generates. Devices are
assigned to subjects by means of device references in the subject specification
part.

Since resource allocation is static, a global mapping of hardware interrupt to
destination subject can be compiled at integration time. The kernel then uses
this mapping at runtime to determine the destination subject that constitutes
the final recipient of the interrupt.

Each subject has a list of pending interrupts. An interrupt is forwarded to a
subject by appending an entry to the interrupt list of the destination subject.
When the execution of a subject is resumed, the kernel consults this list and
injects the interrupt completing its delivery.

Spurious or invalid interrupts that have no valid interrupt to subject mapping
are ignored by the kernel.

\subsection{Multicore}\label{subsec:multicore}
TODO: synchronization, symmetric, per-cpu storage concept, major frame index
consistency

\subsection{Scheduling}\label{subsec:scheduling}
This section presents the design of the Muen kernel scheduler and the selected
scheduling algorithm.

In the context of this work, scheduling is defined as the process of selecting
a subject and giving it access to system resources for a certain amount of time.
The main resource is processor time, which enables a subject to execute and
perform its task.

A key objective of the scheduler is to provide temporal isolation of all
subjects. In order to meet this requirement, the scheduler must prevent any
interference between subjects. To achieve this, scheduling is done in a fixed,
cyclic and preemptive manner.

Subjects are executed for a fixed amount of time, before being preempted by the
scheduler. Preemption means that regardless of what operations a subject is
performing, its execution is suspended when the alloted time quantum has been
consumed. After a subject has been suspended, the scheduler executes the next
subject for a given amount of time.

The information of which subject runs for how long and the chronological
sequence of subjects is declared in a \emph{scheduling plan}\index{Scheduling
plan}.  Such a plan is part of the policy and specifies in what order subjects
are executed on which logical CPU and for how long. The task of the scheduler is
to enforce a given scheduling regime.

A scheduling plan is specified in terms of frames. A \emph{major frame}
\index{Major frame} consists of a sequence of minor frames. When the end of a
major frame is reached, the scheduler starts over from the beginning and uses
the first minor frame in a cyclic fashion. This means that major frames are
repetitive. A \emph{minor frame}\index{Minor frame} specifies a subject and a
precise amount of time. This information is directly applied by the scheduler.

Figure \ref{fig:example-major-frame} illustrates the structure of a major frame.
The major frame consists of four minor frames. Minor frame two has twice the
amount of ticks than the other minor frames, which have identical length.

When the major frame starts, subject one is scheduled for the length of minor
frame one, followed by a switch to subject 2. After that the two subjects are
again scheduled in alternating fashion.

\begin{figure}[ht]
	\begin{ganttchart}[
		vgrid={*3{dotted},*1{dashed},*7{dotted},*1{dashed},*3{dotted},*1{dashed},*3{dotted}},
		hgrid,
		y unit title=0.75cm,
		title label anchor/.style={below=-1.5ex}]{20}
		\gantttitle{Major frame}{20} \\
		\gantttitle{Minor 1}{4}
		\gantttitle{Minor 2}{8}
		\gantttitle{Minor 3}{4}
		\gantttitle{Minor 4}{4} \\
		\ganttbar[bar/.append style={fill=CornflowerBlue}]{Subject 1}{1}{4}
		\ganttbar[bar/.append style={fill=CornflowerBlue}]{}{13}{16} \\
		\ganttbar[bar/.append style={fill=YellowGreen}]{Subject 2}{5}{12}
		\ganttbar[bar/.append style={fill=YellowGreen}]{}{17}{20}
	\end{ganttchart}
	\caption{Example major frame}
	\label{fig:example-major-frame}
\end{figure}

An example scheduling plan for multiple logical CPUs is depicted in figure
\ref{fig:example-scheduling-plan}. It illustrates a system with two logical CPUs
that execute various subjects indicated by different colors.

\begin{figure}[ht]
	\begin{ganttchart}[
		vgrid={*9{dotted},*1{dashed},*9{dotted}},
		hgrid,
		y unit title=0.75cm,
		title label anchor/.style={below=-1.5ex}]{20}
		\gantttitle{Major frame 1}{10}
		\gantttitle{Major frame 2}{10} \\
		\ganttbar[bar/.append style={fill=Apricot}]{CPU0}{1}{10}
		\ganttbar[bar/.append style={fill=Apricot}]{}{11}{20} \\
		\ganttbar[bar/.append style={fill=CornflowerBlue}]{CPU1}{1}{2}
		\ganttbar[bar/.append style={fill=YellowGreen}]{}{3}{6}
		\ganttbar[bar/.append style={fill=CornflowerBlue}]{}{7}{8}
		\ganttbar[bar/.append style={fill=YellowGreen}]{}{9}{10}
		\ganttbar[bar/.append style={fill=CornflowerBlue}]{}{11}{12}
		\ganttbar[bar/.append style={fill=YellowGreen}]{}{13}{16}
		\ganttbar[bar/.append style={fill=CornflowerBlue}]{}{17}{18}
		\ganttbar[bar/.append style={fill=YellowGreen}]{}{19}{20}
	\end{ganttchart}
	\caption{Example scheduling plan}
	\label{fig:example-scheduling-plan}
\end{figure}

CPU0 is executing the same subject for the whole duration of the major frame.
This could for example be the $\tau$0 subject executing on the bootstrap
processor (BSP). The second CPU is executing two subjects (blue and green) in
alternating order. As can be seen, subject green is granted more CPU cycles than
subject blue. Since major frames are repeated, major frame one and two are
identical. All CPUs of the system wait on a barrier at the beginning of a new
major frame.  This guarantees that all logical CPUs of a system are in-sync when
major frames change.

On systems with multiple logical CPUs, a scheduling plan must specify a sequence
of minor frames for each processor core. In order for the cores to not run out
of sync, a major frame must be of equal length on all CPUs. This means that the
sum of all minor frame time slices of a major frame must amount to the same time
duration.

Since a system performs diverse tasks with different resource requirements,
there is a need for some flexibility with regards to scheduling. To provide this
degree of freedom while keeping the kernel complexity low, multiple scheduling
plans can be specified in the system policy. The privileged subject $\tau$0 is
then allowed to select and activate one of these scheduling plans. The kernel
enforces the current plan designated by $\tau$0.

By defining a distinct plan for each anticipated workload in the policy, the
scheduling regimes are fixed at integration time. This removes any runtime
uncertainty that might be introduced by this enhancement, since the scheduling
plans cannot be altered at runtime.
