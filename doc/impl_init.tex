\subsection{Init}\label{subsec:init}
After reset of a x86 system, the processor begins executing code at physical
address \texttt{ffff:0000}, which is mapped to the PC BIOS\index{BIOS}. The
BIOS first performs tests and initialization routines and then searches for a
bootable storage medium. If found, the BIOS copies the first sector of the
storage media to physical address \texttt{0000:7C00} and jumps to this address
(i.e. starts executing code at this address). This is where the system
bootloader, responsible to boot operating systems according to its
configuration, initially starts execution. Many bootloaders first load
additional code from the storage media prior to preparing the system
environment for OS execution.

The Muen separation kernel is compliant to the multiboot specification, version
0.6.96 \cite{multiboot}. The multiboot standard is used to uniformly boot
different operating system kernels by multiboot-aware bootloaders.  The Muen
kernel exports the required multiboot header within the first 8192 bytes of the
OS image. The bootloader loads the OS image into memory and jumps to the
physical kernel entry point according to the information contained in the
header.

\begin{figure}[h]
	\centering
	\begin{bytefield}{24}
	\bitbox[]{10}{}
	\bitbox[]{16}{$\vdots$}\\[1ex]
	\memsection{0010 0000}{0020 3fff}{6}{Kernel memory}\\
	\memsection{0000 8000}{000f ffff}{3}{\color{Gray}-- free --}\\
	\begin{rightwordgroup}{VMX regions}
		\memsection{0000 5000}{0000 7fff}{2}{\color{Red}VMCS}\\
		\memsection{0000 1000}{0000 4fff}{2}{\color{Red}VMXON}
	\end{rightwordgroup}\\
	\memsection{0000 0000}{0000 0fff}{2}{\color{Red}AP trampoline}
\end{bytefield}

	\caption{Example memory layout on system init}
	\label{fig:init-mem-layout-example}
\end{figure}

Figure \ref{fig:init-mem-layout-example} shows the physical memory layout of an
example system. The kernel entry point of this system is at physical address
\texttt{0x100000}. The kernel uses the memory region starting from this address
to physical address \texttt{0x203fff} for code and data.

Physical memory below 1 MB, called low-memory, is used for system data
structures (colored in red). The AP\index{AP} trampoline\index{trampoline} is
needed to bootstrap the system's application processors as described in the
following section. It initially resides in the kernel's text section and must
be copied to low-memory by the init code.

To enable VMX operation using the \texttt{VMXON} instruction, one page per
logical CPU is required. This region is called VMXON\index{VMXON} region. Each
subject is managed using a VMCS\index{VMCS} data structure, so this memory is
placed in low-memory as well. The physical location of these regions is
specified in the policy.

It is the bootloader's task to prepare the system state as dictated by the
multiboot standard, see \cite{multiboot} section 3.2 for details. The kernel
can expect the system to be in this exact state. Before the Muen kernel jumps
into the main SPARK code, it performs additional preparatory steps. This
initial startup code is written in Assembly and conducts the following tasks:

\begin{enumerate}
	\item Copy the AP trampoline to low-memory, see section
		\ref{subsec:mp-support} \item Initialize per-CPU VMXON regions
	\item Initialize subject VMCS regions
	\item Enable Physical Address Extensions\footnote{PAE is a prerequisite for enabling IA-32e mode} (PAE\index{PAE})
	\item Initialize per-CPU kernel page tables
	\item Enable IA-32e mode and execute-disable (NX)
	\item Enable paging, write protection, caching and native FPU error
		reporting
	\item Set up 64-bit Global Descriptor Table\footnote{GDT is a data structure used to define memory segments and their properties} (GDT\index{GDT})
	\item Set up Page-Attribute Table (PAT)
	\item Set up kernel stack
	\item Initialize Ada runtime
	\item Jump into SPARK main procedure
\end{enumerate}
