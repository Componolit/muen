\chapter{Background}
This chapter introduces technologies and concepts which are necessary for the
understanding of the design and the realization of the Muen kernel.  First we
present SPARK, the programming language chosen for the implementation.  After
that, a short description of the Intel/PC architecture is given, followed by an
introduction into virtualization. The concept of the separation kernel is
presented and the main motivation and goals of this work is laid out. The
chapter concludes with an overview of related projects.

\section{SPARK}\label{sec:spark}
SPARK\footnote{Short for SPADE Ada Kernel} is a formally-defined high-level
programming language designed for implementing high integrity systems. It is
based on Ada, which is itself a programming language with a strong focus on
security and safety.

The SPARK language is a subset of Ada with additional features inserted as
annotations in the form of Ada comments. Since compilers ignore comments and
SPARK is a true subset of Ada, any correct SPARK program is a correct Ada
program and can be compiled using existing Ada compilers. One such compiler is
GNAT, which is part of the GNU compiler collection (GCC) \cite{gcc}.

However, since annotations are an integral part of SPARK, it would be
misleading to simply consider SPARK a constrained version of Ada. SPARK should
be viewed as a programming language in its own right. The following list
summarizes the Ada restrictions imposed by SPARK:

\begin{itemize}
	\item No access types (pointers)
	\item No recursion
	\item No exceptions
	\item No goto
	\item No anonymous and aliased types
	\item No functions with side-effects
	\item No dynamic array bounds
	\item No dynamic memory allocation
	\item No dynamic dispatching
\end{itemize}

Annotations are processed by SPARK tools. These tools perform static analysis of
source code. The annotations allow the tools to do data and information flow
analysis as well as prove the absence of runtime errors. This means that SPARK
tools allow one to formally verify that a given program is free of errors such
as division by zero, out-of-bounds array access etc. By using SPARK the
following types of errors can be proven to be absent from the code:

\begin{itemize}
	\item Incorrect indexing of arrays
	\item Overflows
	\item Division by zero
	\item Type range violations
	\item Memory exhaustion
	\item Dangling pointers
\end{itemize}

On top of these properties, the usage of pre-/post-conditions and assertions
allow to prove additional functional properties. A proof of (partial
\footnote{Termination cannot be shown}) correctness of SPARK programs is
achievable. This allows one to formally show the correspondence of an
implementation with a formal specification.

It is also interesting to note, that SPARK has support for tasking in the form
of a language profile\footnote{Restricted subset of the programming language}
called RavenSPARK \cite{RavenSPARK}.

SPARK is a mature technology and has garnered quite some interest since it has
been successfully used in several industrial projects
\cite{Chapman:2000:IES:369264.369270}. It is primarily employed in the field of
avionics, space, medical systems and the defense industry.

\subsection{Design rationale}
The main driving factors behind the design of SPARK are briefly described here:

\begin{description}
	\item[Logical soundness] \hfill \\
		The language must not contain any ambiguities and must be formally
		defined.
	\item[Complexity of formal language definition] \hfill \\
		The language must be simple to specify formally.
	\item[Expressive power] \hfill \\
		The language must be expressive enough to implement complex systems.
	\item[Security] \hfill \\
		All language rules must be statically checkable with reasonable effort
		(i.e. in polynomial time).
	\item[Verifiability] \hfill \\
		Program verification must be tractable for industrial scale projects.
	\item[Bounded space and time requirements] \hfill \\
		Resource requirements must be determined statically.
\end{description}

\subsection{Example}
The following listing illustrates how annotations are used to specify the
contract of a subprogram.

\begin{lstlisting}[language=Ada]
type Color_Type is (Red, Green, Blue);

procedure Exchange (X, Y: in out Color_Type);
--# derives X from Y &
--#         Y from X;
--# post X = Y~ and Y = X~;
\end{lstlisting}

The declaration of the \texttt{Exchange} procedure states that it has two
parameters X and Y which are of mode \texttt{in out}. This means that the values
of both parameters are imported (\texttt{in}) and exported (\texttt{out}).

Since the specification does not contain a \texttt{global} annotation, no other
state (i.e. global variables) is accessed and the procedure is therefore free of
side-effects. The \texttt{derives} annotation states the data flow: the value of
X is derived from Y and vice versa.

Lastly, the postcondition specifies the values of X and Y after the procedure
has been executed. An identifier decorated with the tilde symbol ($\sim$)
indicates the initial imported value of the variable. Thus the \texttt{post}
annotation states that X will be assigned the initial value of Y and likewise
for Y.

It should be noted that the \texttt{Color\_Type} is a \emph{distinct}
enumeration type which \emph{cannot} be mixed with other types.

An in-depth discussion of the SPARK programming language can be found in
\cite{BarnesSPARK}.

\subsection{SPARK 2014}
At the time of writing\footnote{August 2013}, the SPARK language is undergoing
a major transformation. The goal is to extend the subset of Ada included in
SPARK and to make use of the new Ada 2012 features \cite{Ada2012}. The use of
Ada 2012 aspects will replace the SPARK annotation comments. The official SPARK
2014 release is expected in the first quarter of 2014
\cite{SPARK2014:Announcement}.

Since the development of SPARK 2014 is currently ongoing, the Muen kernel is
implemented using the existing SPARK 2005 language and tools.

\section{Intel x86 Architecture}
Even though a more detailed familiarity with the subject is required for a full
understanding of the design and implementation of the Muen kernel, this section
tries to give a short tour of the Intel x86 architecture. The interested reader
is directed to the Intel Software Developer's Manuals (SDM\index{SDM})
\cite{IntelSDM}, which contain a complete and in-depth description of the x86
architecture.

The basic components of a modern Intel x86 computer are depicted in figure
\ref{fig:intel-architecture}.

\begin{figure}[h]
	\centering
	\begin{tikzpicture}
	\node[apribox, minimum width=3cm, minimum height=1cm] (cpu) {CPU};
	\node[bluebox] (gfx) [left=of cpu] {PCIe/Graphics Adapter};
	\node[bluebox] (ram) [right=of cpu] {RAM};

	\node[commonbox, minimum width=3cm, minimum height=1cm] (pch) [below=1cm of cpu] {PCH};
	\node[greenbox] (network) [left=of pch] {Network};
	\node[greenbox] (usb) [below left=of pch] {USB};
	\node[greenbox] (pci) [below=of pch] {PCI Express};
	\node[greenbox] (bios) [below right=of pch] {BIOS};
	\node[greenbox] (legacy) [right=of pch] {Legacy};

	\draw[thick] (cpu) to node[auto] {DMI} (pch);
	\draw[thick] (cpu) -- (gfx);
	\draw[thick] (cpu) -- (ram);
	\draw[thick] (pch) -- (network);
	\draw[thick] (pch) -- (usb);
	\draw[thick] (pch) -- (pci);
	\draw[thick] (pch) -- (bios);
	\draw[thick] (pch) -- (legacy);
\end{tikzpicture}

	\caption{Intel architecture}
	\label{fig:intel-architecture}
\end{figure}

\subsection{Processor}
The main processor of the system is called the central processing unit
(CPU\index{CPU}). Multi-processor systems (MP\index{MP}) have multiple CPUs
which in term can have multiple cores. Each core in each CPU is a so called
\emph{logical CPU} which executes instructions. These instructions change the
state of the logical CPU and the system as a whole.

\subsubsection{Execution environment}\label{subsubsec:exec-env}
A program executing on an processor is given access to resources to store and
retrive information, such as code or data. Resources provided by a modern
64-bit processor constituting the basic execution environment are described in
the following list:

\begin{description}
	\item[Address space] is used by a program to access system memory.
	\item[Stack] is located in memory and facilitates subprogram calls and
		parameter passing in conjunction with stack management resources of the
		processor.
	\item[Execution registers] constitute the core of the execution environment.
		It consists of sixteen general-purpose, six segment and one flag
		register as well as the instruction pointer.
	\item[Control registers] define the current processor operating mode and
		other properties of the execution environment.
	\item[Descriptor table registers] are used to store the location of memory
		management and interrupt handling data structures.
	\item[Debug registers] allow a program to control and use the debugging
		functionality of a processor.
	\item[x87 FPU registers] offer support for floating point operations.
	\item[MMX registers] offer support for single-instruction, multiple-data
		(SIMD) operations for integer numbers.
	\item[XMM registers] offer support for SIMD operations on floating-point
		numbers.
	\item[Model-specific registers (MSRs)] provide control of various hardware
		and software-related features. Their number and functionality varies
		depending on the given processor implementation.
	\item[I/O ports] allow transfer of data to and from Input/Output ports.
\end{description}

Further details about the Intel processor execution environment can be found in
\cite{IntelSDM}, volume 1, section 3.2.1. All instructions provided by a
processor are called an instruction set. The complete Intel instruction set
architecture (ISA)\index{ISA} is specified in \cite{IntelSDM}, volumes 2A-C.

\subsubsection{Caches}
The processor has numerous internal caches and buffers of varying sizes and
properties. Their main purpose is to increase processor performance by
optimizing accesses, e.g. reading data from system memory. The most important
caches are outlined in the following list:

\begin{itemize}
	\item Level 1 instruction cache
	\item Level 1 data cache
	\item Level 2 \& 3 unified caches
	\item Translation lookaside buffers (TLB)
	\item Store buffer
	\item Write Combining buffer
\end{itemize}

Since caches are shared and because they can only be controlled to a limited
degree, their state can be changed and observed by different parts of a system.
Thus some of the caches can be used as side-/covert-channels and pose a
challenge to effective component isolation.

\subsubsection{Protected mode}
A modern CPU provides several operating modes of which one is called
\emph{protected mode}. In this mode the processor provides different privilege
levels also called rings. The rings are numbered from 0 to 3, with ring 0 having
the most privileges. Due to common operating systems using only rings 0 and 3
while disregarding other privilege levels, the 64-bit extension to the Intel
architecture dropped support for rings 1 and 2.

Privileged instructions such as switching memory management data structures are
only executable in ring 0, also called supervisor mode. Operating systems such
as Linux, usually execute user applications in the unprivileged ring 3, called
user mode.

\subsubsection{IA-32e mode}
Intel 64 architecture runs in a processor mode named \emph{IA-32e} also known as
\emph{long mode}. It has two submodes: compatibility and 64-bit mode. The first
submode allows the execution of most 16 and 32-bit applications to run without
re-compilation by essentially providing the same execution environment as 32-bit
protected mode.

The IA-32e 64-bit submode enables 64-bit kernels and operating systems to run
applications making use of the full 64-bit linear address space. The execution
environment is extended by additional general purpose registers and most
register sizes are extended to 64-bit.

In the context of this project, IA-32e mode generally refers to the 64-bit
IA-32e submode if not stated differently. Additional information about this mode
of operation and the provided execution environment can be found in
\cite{IntelSDM}, volume 1 section 3.2.1.

\subsection{Memory management}
Current processors support management of physical memory through means of
segmentation and paging. Logical addresses are translated to linear addresses
using the segmentation mechanism. In a second step, linear addresses are
translated to physical addresses using paging.

Memory management is done in hardware by the memory management unit
(MMU)\index{MMU}.  An operating system must set up certain data structures
(descriptor tables and page tables) to instruct the MMU how logic and linear
memory addresses map to physical memory.

Processes running on a processor are provided with a virtual address space. The
processes think they run alone on a system and have unrestricted access to
system memory. Memory access of processes are translated by the MMU using the
aforementioned paging structures.

Logical addresses are transformed to linear addresses by adding an offset to the
given virtual address. This mechanism is called segmentation. In IA-32e
segmentation has effectively been dropped in favor of paging, creating a flat
64-bit linear address space.

Paging is the address translation process of mapping linear to physical
addresses. Memory is organized in so called pages. A memory page is the unit
used by the MMU to map addresses. IA-32e mode provides different page sizes
such as 4 KB, 2 MB and 1 GB.

\begin{figure}[h]
	\centering
	\begin{tikzpicture}[minimum height=0.6cm]
	\node[bluebox] (pti) {Index};
	\node[bluebox, minimum width=3cm, right=1mm of pti] (ofs) {Offset};
	\node[above=1mm of ofs, xshift=-0.5cm] (via) {Virtual address};

	\node[greenbox, minimum width=3.2cm, minimum height=1.5cm, label={[yshift=-18pt]Page table}, below=of ofs] (pta) {};
	\node[graybox] (pte) at (pta.south) [above=5pt] {Page table entry};

	\node[apribox, minimum width=3.2cm, minimum height=1.5cm, label={[yshift=-18pt]Page frame}, right=of pta, xshift=-0.5cm, yshift=1.5cm] (php) {};
	\node[graybox] (paf) at (php.south) [above=5pt] {Physical address};

	\draw[arrow] (pti) |- (pte);
	\draw[arrow] (ofs) |- (paf);
	\draw[arrow] (pte) -| (php);
\end{tikzpicture}

	\caption{One-level address translation}
	\label{fig:address-translation}
\end{figure}
TODO:Draw figure

An exemplary one-level address translation process illustrated in
figure \ref{fig:address-translation}. The MMU splits the linear address in
distinct parts which are used as indexes into page tables. A page table entry
specifies the address of a physical memory page called \emph{page frame}.
Addition of the offset part of the input address to the page frame yields the
effective physical address.

Paging structures can be arranged hierarchically by letting page table entries
reference other paging structures instead of physical memory pages. In fact,
Intel's IA-32e mode uses four levels of address mapping which are presented in
the following list:

\begin{itemize}
	\item Page map level 4 (PML4) references a page directory pointer table. The
		address of the currently active PML4 is stored in the CPU's CR3 control
		register.
	\item Page directory pointer table (PDPT) references a page directory or a 1
		GB page frame.
	\item Page directory (PD) references a page table or a 2 MB page frame.
	\item Page table (PT) references a 4 KB page frame.
\end{itemize}

A linear memory address is split into five parts, when all four levels of paging
structures are used. The first element identifies the PML4 entry, the second
references the PDPT entry and so on. The fifth part constitutes the offset added
to the page frame address. Larger page sizes are supported by letting higher
level paging structures reference physical memory pages. In such a case all
remaining parts of the linear memory address are used as offset into the larger
page frame.

Additionally paging structure entries allow to specify properties and
permissions such as write access or caching behavior for the referenced physical
memory page. The permissions are checked and enforced by the MMU. This mechanism
allows for fine-grained memory management on a per-process basis. The use of
different paging structures depending on the currently executing process allows
for partitioning and separation of system memory using the MMU. This can be
achieved by changing the value of the processor's CR3 control whenever a process
switch occurs.

If a process tries to access a memory location that has no valid address
translation the processor raises a page fault exception. The operating system's
page fault handler is then in charge to correct the failure condition, e.g.
dynamically allocating and mapping the requested memory region by changing the
paging structures. It can then seamlessly resume the execution of the faulting
process. This technique is used by many modern operating systems such as Linux
to implement dynamic memory management.

A processor is connected to the memory controller hub via the system bus, which
is also called front side bus (FSB)\index{FSB}.

For a more in-depth look at memory management and address translation the reader
is referred to \cite{IntelSDM}, volume 3A sections 3 \& 4 and
\cite{Drepper07whatevery}.

\subsection{Exceptions and interrupts}
Exceptions and interrupts signal an event that occured in the system or the
processor that needs attention. They can occur at any time and are normally
handled by preempting the currently running process and transfering execution to
a special interrupt service routine (ISR).

Each exception or interrupt has an associated number, in the range of 0 to 255,
which is called the interrupt vector. Numbers in the range of 0 to 31 are
reserved for the Intel architecture. They are used to uniquely identify all
specified exceptions, see \cite{IntelSDM}, volume 3A section 6.15. The remaining
vectors can be freely used by hardware devices and the operating system.

Interrupts are caused by hardware devices to notify the processor that the
device needs servicing. An example is a network card that generates an interrupt
whenever a data packet is received from the network. The interrupt handler
responsible for servicing the network card is invoked upon recognition of the
event and can then copy the data to system memory and acknowledge the successful
data transfer to the device.

Exceptions are generated by the processor itself when it detects an error
condition during instruction execution. Causes for exceptions are for example
division by zero or page faults.

Interrupts generated by external devices can be blocked by disabling the CR0.IF
flag. When the flag is not set, such interrupts are not recognized by the
processor until the flag is enabled again. Exceptions however are not affected
by the CR0.IF flag and are processed as usual.

The main data structure which facilitates interrupt handling is the interrupt
descriptor table (IDT)\index{IDT}. It is a list of entries which point to
interrupt handler procedures. The IDT can be located anywhere in system memory
and its physical address must be stored in the IDT registers (IDTR). When the
processor receives and interrupt, it uses the interrupt vector as an index into
the IDT to determine the associated handler. Execution control is then
transfered by invoking the handler procedure.

\subsection{Memory Controller Hub}
The memory controller hub (MCH)\index{MCH}, also called
\emph{Northbridge}\index{northbridge}, connects the CPU to random access memory
(RAM), the graphics card and southbridge (presented in the next section) of the
system. The processor accesses these resources by sending memory requests via
the system bus. The MCH decodes the requests and forwards them to RAM or the
Southbridge, depending on the physical address. If for example a request
contains an address of a memory-mapped device, the chipset of the MCH forwards
the request to the Southbridge.

\subsection{Input/Output Controller Hub}\label{subsec:ich}
The input/ouput controller hub (ICH\index{ICH}), also called
\emph{Southbridge}\index{southbridge}, provides connections to most devices and
platform peripherals, such as keyboards, USB connections and other buses (PCI
Express etc). Newer models contain an I/O memory management unit
(IOMMU\index{IOMMU}), which provides address translation functionality similar
to the MMU, but for accesses to system memory initiated by I/O devices.  This is
necessary since PCI devices perform direct memory access (DMA)\index{DMA}, which
bypasses the processor's MMU memory protection.  Like the MMU, an operating
system must initialize data structures and setup the IOMMU for proper separation
of physical memory accessible by devices.

\subsection{Platform Controller Hub}
More recently, Intel has rearranged the presented hub architecture and has
integrated much of the functionality provided by the Northbridge into the CPU
\cite{IntelQPI}. All remaining functions of the Northbridge and Southbridge have
been combined and incorporated into a new component called the platform
controller hub (PCH)\index{PCH}. This new architecture avoids the bottleneck
presented by the system bus connection between the CPU and the Northbridge.

\subsection{Programmable Interrupt Controller}\label{subsec:apic}
A Programmable Interrupt Controller (PIC\index{PIC}) is used to connect several
input sources to a CPU. Hardware devices raise interrupts to inform a CPU that
some event occurred which must be handled. For example a network card notifies
the CPU that it received data which must be processed. One of the best known PIC
is the Intel 8259A which was part of the original PC\footnote{Personal
computer}\index{PC} introduced in 1981.

Early PC/XT ISA\index{ISA} systems used one 8259 controller, allowing only eight
interrupt input lines. By cascading multiple controllers, more lines can be made
available but nevertheless, a more flexible approach was needed with the advent
of multiprocessor (MP\index{MP}) systems containing multiple processor cores.
Implementing efficient interrupt routing on an MP system using PICs was an
issue.

Intel presented the Advanced Programmable Interrupt Controller
(APIC\index{APIC}) concept with the introduction of the Pentium processor. It
was one of several attempts to solve the interrupt routing efficiency problem.
The Intel APIC system is composed of two components: a local APIC
(LAPIC\index{LAPIC}) in each CPU of the system and the I/O APIC, see figure
\ref{fig:apic} for a schematic view.

\begin{figure}[h]
	\centering
	\begin{tikzpicture}
	\node (du8) {};
	\node[bluebox, minimum width=2cm, minimum height=1.4cm, label={[yshift=-18pt]CPU0}, left=5mm of du8] (cp0) {};
	\node[graybox] (la0) at (cp0.south) [above=5pt] {LAPIC};
	\node[bluebox, minimum width=2cm, minimum height=1.4cm, label={[yshift=-18pt]CPU1}, right=5mm of du8] (cp1) {};
	\node[graybox] (la1) at (cp1.south) [above=5pt] {LAPIC};

	\node[align=center, left=8mm of la0]  (li0) {Local\\Interrupts};
	\node[align=center, right=8mm of la1] (li1) {Local\\Interrupts};

	\node[inner sep=0, below=6mm of cp0] (du0) {};
	\node[inner sep=0, below=6mm of cp1] (du1) {};

	\draw[arrow] (li0) -- (la0);
	\draw[arrow] (li1) -- (la1);
	\draw[thick, <->] (cp0) -- (du0);
	\draw[thick, <->] (cp1) -- (du1);
	\draw[very thick] (du0) to node[auto, name=sys] {System Bus} (du1);

	\node[apribox, below=6mm of sys]  (ioa) {I/O APIC};
	\node[align=center, right=of ioa] (ext) {External\\Interrups};

	\draw[thick, <->] (ioa) -- (sys);
	\draw[arrow] (ext) -- (ioa);
\end{tikzpicture}

	\caption{Local APICs and I/O APIC}
	\label{fig:apic}
\end{figure}

A LAPIC receives interrupts from internal sources (such as the timer interrupt)
and from the I/O APIC or other external interrupt controllers. It sends these
interrupts to the processor core for handling. In MP systems the LAPIC also
sends and receives inter-processor interrupt messages (IPI\index{IPI}) from and
to other logical processors on the system bus.

The I/O APIC is used to receive external interrupts from the system and its
associated devices and relay them to the local APICs. The LAPICs are connected
to the I/O APIC via system bus. The routing of interrupts to the appropriate
LAPICs is configured using a redirection table which must be setup by the
operating system. For more information about APIC and I/O APIC see chapter 10 in
\cite{IntelSDM}.

\section{Virtualization}
Virtualization is an established architectural concept in computer science and
has been in use for decades. Operating systems for example provide virtual
address spaces to application processes, giving them the illusion of unlimited,
continuous memory. Hence an application does not need to take care of complex
memory management tasks and the operating system is able to optimize the usage
of physical memory. Another common example is the virtualization of devices such
as virtual CD-ROM drives directly using a file-based backend for data I/O
\cite{CryptoCloud}.

Hardware or platform virtualization is the process of simulating virtual
computer hardware that acts like real hardware. The virtualization is performed
and controlled by special software, called a
\emph{hypervisor}\index{hypervisor} or \emph{virtual machine monitor}
(VMM\index{VMM}). These two terms are synonyms; for consistency we will use VMM
throughout this document. VMMs are classified into two types, as shown in figure
\ref{fig:vmm-classification}.

\begin{figure}
	\centering
	\begin{subfigure}[b]{0.24\textwidth}
		\centering
		\begin{tikzpicture}[text width=2cm, minimum height=0.5cm]
	\node (haw) [apribox]                     {Hardware};
	\node (vmm) [bluebox, above=0.1cm of haw] {VMM};
	\node (vim) [greenbox, above=of vmm]      {Virtual Machine};

	\draw[arrow] (vmm) to (vim);
\end{tikzpicture}

		\caption{\emph{Type I, native or bare metal VMM.} Runs directly on the
		hardware in the most privileged processor mode.}
	\end{subfigure}
	\qquad
	\begin{subfigure}[b]{0.24\textwidth}
		\centering
		\begin{tikzpicture}[text width=2cm, minimum height=0.5cm]
	\node (haw) [apribox]                     {Hardware};
	\node (hos) [apribox, above=0.1cm of haw] {Hosting OS};
	\node (vmm) [bluebox, above=0.1cm of hos] {VMM};
	\node (vim) [greenbox, above=of vmm]      {Virtual Machine};

	\draw[arrow] (vmm) to (vim);
\end{tikzpicture}

		\caption{\emph{Type II or hosted VMM.} The VMM runs on top of a
		conventional operating system and uses OS services.}
	\end{subfigure}
	\caption{VMM classification}
	\label{fig:vmm-classification}
\end{figure}

The VMM runs on the \emph{host} and software that runs in a virtualized
environment is called \emph{guest} software. The terms host and guest are
generally used to distinguish the software that runs on the physical machine
from the software that runs on the virtual machine (VM\index{VM})
\cite{wiki:virtualization}.

By using virtualization and scheduling techniques, a VMM multiplexes the
hardware of a computer making it possible to run multiple guests
simultaneously\footnote{On multicore machines in parallel, on single-core
machines in pseudo-parallel} on a single physical machine. This is the most
common use-case for virtualization: the consolidation of operating system
instances on one server to save power and to improve hardware-resource
utilization.

A guest program is separated from the real hardware of a machine. Direct access
is restricted and can be controlled by the host VMM. Since the VMM has complete
control over the hardware and guest software state, virtualization is also
useful for separation purposes. The VMM is able to allow certain communication
channels between guest software while restricting others. One guest could have
access to the network card, while others share a page in memory but have no
access to hardware devices.

The principle idea of virtualization is to run guest code on a (virtual) CPU
and only intercept privileged operations accessing resources or system
properties for which direct access is prohibited. Various techniques exist to
intercept guest instructions, ranging from inspection and modification of the
guest software instruction stream (slow) to hardware-assisted virtualization
(fast). This thesis focuses on the hardware-assisted approach. The reader is
directed to \cite{VMware:virtualization} for further information about
virtualization mechanisms.

Independent of the chosen mechanism, interception of the running guest leads to
a so called \emph{trap}\index{trap} from the guest code into the VMM. The VMM
then examines the \emph{exit reason} and reacts accordingly by for example
modifying the guest software state and then resuming guest execution.

This method is also used to implement a technique called "trap and emulate".
For example, if a guest accesses a device which is emulated by the VMM, a trap
into the VMM occurs. The VMM itself or a specialized VM monitor emulates the
requested operations of the guest software by directly modifying the state of
the virtual processor or memory of the trapping VM.

With hardware-assisted virtualization, traps are handled by the virtualization
hardware automatically. The exact behavior is configurable by the VMM software.

\subsection{Intel Virtualization Technology (VT)}
A trap from the guest into the VMM is a costly operation. To reduce traps,
modern processors have introduced mechanisms to support the VMM in creating a
virtual machine environment. These features not only improve the performance of
a virtual machine by avoiding traps, but also greatly simplify the VMM
implementation.

Intel Virtualization Technology (VT\index{VT}) provides hardware-assisted
virtualization mechanisms for Intel processors. Intel VT encompasses multiple
virtualization features:
\begin{itemize}
	\item Intel VT-x
	\item Intel EPT
	\item Intel VT-d
\end{itemize}

\subsubsection{Intel VT-x}
Intel VT-x\index{VT-x} provides a virtual machine architecture to allow
efficient processor virtualization. An Intel processor reports this feature
with the Virtual Machine Extensions (VMX\index{VMX}) CPU flag.

The virtual machine architecture is implemented by a new form of processor
operation called VMX operation. This mode is enabled by executing the
\texttt{VMXON} instruction and provides two operating modes: VMX root operation
and VMX non-root operation.

\begin{figure}[h]
	\centering
	\begin{tikzpicture}
	\node[bluebox] (vmm) {VMM};
	\node[left=2.5cm of vmm] (du1) {};
	\node[right=2.5cm of vmm] (du2) {};
	\node[gray!80, font=\scriptsize, above=2.7cm of vmm] (du3) {VMX non-root};

	\node[gray!80, font=\scriptsize, below=2mm of vmm] {VMX root};

	\node[greenbox, left=of du3]  (gu0) {Guest 0};
	\node[greenbox, right=of du3] (gu1) {Guest 1};

	\draw[arrow] (du1) to node[auto] {VMXON}  (vmm);
	\draw[arrow] (vmm) to node[auto] {VMXOFF} (du2);

	\draw[arrow] (vmm) to node[right] {VM Entry} (gu0);
	\draw[arrow] (gu0) to[bend right] node[left] {VM Exit} (vmm);
	\draw[arrow] (vmm) to (gu1);
	\draw[arrow] (gu1) to[bend left] node[right] {VM Exit} (vmm);
\end{tikzpicture}

	\caption{Interaction of VMM and Guests}
	\label{fig:vmm-lifecycle}
\end{figure}

The VMX root operation mode is used by the VMM software while guest software
runs in VMX non-root operation mode. A transition from VMX root to VMX non-root
is called a VM entry, while a transition from VMX non-root to VMX root is called
a VM exit or trap. Figure \ref{fig:vmm-lifecycle} shows the life cycle of a VMM
and the transitions between guest software and the VMM.

Processor behavior in VMX root mode is very similar to non-VMX operation.  The
main difference is the extra instruction set provided by VMX, containing
virtualization-specific instructions used to manage VMX processor mode and
virtual machine control structures.

In VMX non-root operation however, the execution environment is restricted to
facilitate virtualization. Privileged or critical instructions cause VM exits
instead of their normal behavior. This allows the VMM to regain control of
processor resources and, depending on the trapping instruction, take
appropriate action such as emulating certain functionality.

The VMM software initiates a VM entry into guest code by executing the
\texttt{VMLAUNCH} and \texttt{VMRESUME} instructions. Only one guest can be
active on a logical processor at any given time. The VMM regains control using
the VM exit mechanism. If an exit occurs, the VMM analyzes the cause of the
exit and acts accordingly. It may resume the guest software or allocate
processor time to a different guest. The VMM exits VMX operation by calling the
\texttt{VMXOFF} instruction.

The Virtual-Machine Control Structure (VMCS\index{VMCS}) is used to control VMX
non-root operation and VMX transitions. Each virtual machine has an assigned
VMCS which allows fine-grained setup of processor behavior in VMX non-root
mode. A VMCS contains fields which can be written by the
\texttt{VMWRITE}\index{VMWRITE} instruction and read via
\texttt{VMREAD}\index{VMREAD} in VMX root mode. The fields can be categorized
into host-state, guest-state, read-only data and control fields.

On VM entry for example, the state of the host is saved automatically into the
subject VMCS by the VMX extensions and restored again on VM exit. On the other
hand, the guest-state area is used to save and restore guest state on VM exit
and VM entry respectively. The read-only data fields provide VMX status
information and the control fields in the VMCS govern VMX non-root operation.

For further information about the VMX processor extensions and VMCS, the reader
is directed to the respective section in the Intel SDM \cite{IntelSDM}, volume
3C, chapters 23 and 24.

\subsubsection{Intel EPT}\label{subsubsec:ept}
Extended Page Table (EPT\index{EPT}) is Intel's implementation of the Second
Level Address Translation (SLAT\index{SLAT}) virtualization technology. It
provides hardware-assisted translation of guest-physical memory addresses to
host-physical addresses. Guest-physical addresses are translated by traversing a
set of EPT paging structures provided by the VMM to produce physical addresses
that are used to access memory.

The EPT paging structure confines the host-physical memory region that a guest
virtual machine is allowed to access, thereby making it possible to safely run
unmodified guest OS memory management code. Access to host-physical memory
outside of the allowed region results in an EPT violation trap.

Without EPT, the burden of translating guest-physical addresses to
host-physical addresses while guaranteeing guest/vmm and guest/guest memory
space separation rests with the VMM. This is a non-trivial task which is highly
inefficient. EPT removes the complexity of manual memory address translations
via shadow page tables from the VMM, making the code simpler while improving
virtualization speed. For more information about EPT see Intel SDM
\cite{IntelSDM}, volume 3C, section 28.2.

\subsubsection{Intel VT-d}\label{subsubsec:vtd}
Virtualization Technology for Directed I/O (VT-d\index{VT-d}) provides hardware
support for I/O-device virtualization. While VT-x provides the support to
virtualize the platform (i.e. the processor), VT-d is used to simplify the
direct assignment of devices to virtual machines by providing direct memory
access (DMA\index{DMA}) and device interrupt remapping functionality. VT-d
provides an IOMMU\index{IOMMU} required to control device DMA.

The VT-d technology is outside the scope of this master thesis, the reader is
directed to \cite{IntelVTd} for a more in-depth discussion.


\section{Separation Kernel}
In a system with high requirements on security, functions needed to guarantee
these requirements must be isolated from the rest of the system and consolidated
in a Trusted Computing Base (TCB)\index{TCB}. To be trusted, this code must be
as minimal as possible to allow formal verification of code correctness. Lampson
et al. \cite{Lampson:1991:ADS:121133.121160} define the TCB of a computer system
as:
\begin{quote}
	A small amount of software and hardware that security depends on and
	that we distinguish from a much larger amount that can misbehave without
	affecting security.
\end{quote}

Code that is part of the TCB is called \emph{trusted} while components that have
no relevance to the guarantee of security properties are called
\emph{untrusted}.

A separation kernel\index{kernel} (SK\index{SK}) can be seen as the fundamental
part of a component-based system since its main purpose is to enforce the
separation of all software components. The concept of separation kernels was
introduced by John Rushby in a paper published in 1981 \cite{rushby1981}:

\begin{quote}
The task of a separation kernel is to create an environment which is
indistinguishable from that provided by a physically distributed system: it
must appear as if each regime is a separate, isolated machine and that
information can only flow from one machine to another along known external
communication lines.
\end{quote}

The separation kernel must therefore guarantee that the components can only
interact according to a well-defined policy while running on the same physical
hardware.

A system policy dictates the partitioning of hardware resources like CPU,
memory or assignment of devices to components. The kernel guarantees this
isolation by emulating a suitable runtime environment for each component,
creating the impression of multiple (virtual) machines. By using modern
virtualization techniques, the kernel is able to delegate certain management
tasks to the hardware. This allows the separation kernel code to be relatively
simple, which is a precondition for formal verification of software.

Because of the simplicity requirement, a system running on top of a separation
kernel is relatively static. The system policy is compiled to a suitable format
at system integration time and cannot change during runtime. This is also the
main difference between the separation kernel concept and microkernels which
provide hardware abstraction layers, advanced mechanisms for inter-process
communication (IPC\index{IPC}) and dynamic resource management. A separation
kernel does not implement such functionality; its sole purpose is to guarantee
component separation according to a policy. Policy authors must make sure that
the system specification is sound and that only intended communication channels
are specified between components.  Of course, this task can be simplified and
aided by applying appropriate support tools.

\subsection{Subjects}
The separation kernel isolates parts of the TCB into multiple
components\index{component} interacting via well-defined interfaces. In this
paper, such components are called \emph{subjects}\index{subject}.

As previously mentioned, only these features absolutely necessary to guarantee
subject separation are present in a separation kernel. Advanced features
required to isolate and virtualize a complex subject are implemented as
dedicated non-privileged subjects, so called \emph{subject monitors}
(SM\index{SM}). A complex subject could be a complete operating system like
Linux\index{Linux}.

To allow more runtime flexibility, a separation kernel could also employ a
dedicated subject to offload management tasks. Such a subject runs in normal
unprivileged mode but is allowed to interact with the SK over a specialized
interface.

\subsubsection{Trust}
In the larger context of a component-based system, some subjects can be
considered to be part of the TCB and must therefore be trusted to operate
according to their specification. Such subjects must be developed with the same
diligence as the separation kernel itself.

As an example: a subject is in charge of data encryption and passes the
encrypted data to a second subject which provides network connectivity. A
security property could be that information must always be encrypted when
transmitted over the Internet. To achieve this, the encryption subject must
ensure that no unencrypted information flows to the network subject. Thus the
encryption subject must be considered part of the TCB and is called a
\emph{trusted} subject since its failure would break the security guarantee.

On the other hand, the network subject is not security critical as long as the
encryption subject works according to its specification. It is therefore not
part of the TCB, which means it is an \emph{untrusted} subject.

\section{Motivation}
As stated in the previous section, software must be simple in order to
establish trust in the correct functioning of the code, either through manual
review or formal verification. Even though the tools to automate formal
verification are progressing fast, the constraints on the software to be
verified are still severe. If the SLOC\footnote{Source lines of
code}\index{SLOC} count and thus complexity of a software project exceeds a
critical threshold\footnote{Depends on the programming language and support
tools}, it is no longer possible to automatically prove integrity and security
requirements or to perform a manual review, because the required effort is just
too big.

This is the reason why common operating system kernels are not suitable for use
as a foundation for high assurance systems that employ a small TCB. The
Linux\index{Linux} kernel currently has over 14 million lines of code. While
functionality can be loaded as modules, Linux is still a
monolithic\index{monolithic} kernel at the architectural level, with the
complete code running in the same address space and privilege level. A
programming error in a device driver for example could compromise the integrity
of the entire system.

While commercial separation kernel products exist on the market, there is
currently no open-source SK\index{SK} available. This is a major drawback for
implementers of a high assurance system since they must trust the most
fundamental building block, the kernel, without the possibility to review the
code or the verification documents (if any). It is of course also impossible to
adapt the kernel code to the specific needs of a customer, which may hinder
development of the overall system.

The motivation for this project is to provide a freely-available SK with a
permissive licensing\index{license} model to allow reviews and modification of
the source code and design documents. It should be possible to independently
reproduce verification artifacts such as automated proofs. In the eyes of the
authors, only this approach is viable in environments where high integrity and
security is demanded.

There have been many recent advances in hardware support for virtualization
that greatly simplify the design and implementation of a security kernel. The
authors believe that now is the perfect time to launch a separation kernel
project which benefits from these hardware capabilities.

\section{Goals}
The main goal of this project is to implement a freely-available, open-source
separation kernel. The kernel sources will be licensed under the GNU General
Public License (GPL\index{GPL}, \cite{gpl}).

SPARK is chosen as the main implementation language since this guarantees the
availability of advanced and approved tools to write high assurance code. The
SPARK tools will be utilized to show certain properties of the kernel code,
especially proof of absence of runtime errors is desired. The code of the kernel
must also be minimal to facilitate a small TCB.

Non-essential operations should be implemented in a special trusted subject
written in SPARK. This subject is called $\tau$0 and is considered an
integral part of the Muen TCB. While keeping the kernel as simple as possible,
this will allow for later adoption of more dynamic resource handling and
scheduling mechanisms.

The target platform of the SK is 64-bit Intel. No abstraction layer will be
provided to support other processor architectures. Such a layer could be added
later as needed. Instead, the kernel must leverage the latest hardware features
of the Intel platform to implement the isolation while maintaining a small code
base.

The main task of the SK is to separate subjects. This is why it must only allow
intended data flows according to the policy and it should prevent or limit
possible side- or covert-channels.

\section{Related work}
This section presents related projects. While they share many similarities, the
main difference is that the Muen kernel aims to combine various approaches: use
of advanced Intel hardware virualization features, Intel x86 64-bit target
platform, minimal code size, application of verification and proof techniques
and making the source code and documentation freely available.

\subsection{seL4}
seL4 is a microkernel of the L4 \cite{Liedtke:1996:TRM:234215.234473} family
which has been formally verified \cite{Klein_EHACDEEKNSTW_09}. It aims to
provide high assurance of functional correctness by means of machine-checked
formal proofs. Using the theorem prover Isabelle/HOL
\cite{Nipkow-Paulson-Wenzel:2002}, it has been shown that the C implementation
correctly implements an abstract specification.

To our knowledge seL4 is the most advanced project in the field of formal
verification combined with operating system research. Unfortunately its
availability and use is rather limited by restrictive licensing. While parts of
the seL4 project are available for non-commercial use, the source code of the
kernel has not been published. Thus it is not possible to reproduce the formal
proofs.

\subsection{XtratuM}
XtratuM is a type 1 hypervisor specially designed for real-time embedded
systems \cite{xtratum:2009a}. It is developed by the Real-Time Systems group of
the Universidad Politécnica de Valencia. It uses para-virtualization to provide
one or more virtual execution environments for partitions. This means
software running as partitions must be modified accordingly to run on top of
XtratuM.

The processor privilege mechanism (supervisor and user mode) is used to
separate the VMM from partitions.

The whole project is open-source and published under the GPL license. It is
implemented in C and Assembly. While it supports various platforms such as
LEON2/3 (PowerPC) and Intel x86, it does not support IA32-e mode.

\subsection{NOVA}
NOVA is a recursive acronym and stands for NOVA OS Virtualization Architecture
\cite{Steinberg:2010:NMS:1755913.1755935}. It applies microkernel construction
principles to create a virtualization environment. Its authors have coined
the term microhypervisor\index{microhypervisor} which is short for
microkernelized hypervisor.

NOVA has been developed from scratch with the goal to achieve a thin hypervisor
layer, an user-level virtual-machine monitor and additional unprivileged
components to reduce the attack surface on the most privileged code and thus
increase the overall system security. It runs on Intel and AMD x86 processors
with support for hardware virtualization and is implemented in the C++
programming language. The source code is publicly available \cite{NOVA} and has
been released under the GPL open-source license.

While NOVA is a very promising architecture it has some drawbacks: the choice of
programming language (C++) and its dynamic nature (in general a very desirable
property) increases the verification complexity to the point where it might be
infeasible.

\subsection{Commercial separation kernels}\label{subsec:commercial-sks}
There are several commercial offerings for separation kernels by various
vendors:

\begin{itemize}
	\item INTEGRITY-178B by Green Hills Software, Inc. is a separation kernel
		EAL-6+ certified against the separation kernel protection profile
		\cite{SKPP}.
	\item PikeOS by SYSGO AG is a microkernel-based real-time OS.
	\item LynxSecure by LynuxWorks Inc. is a type 1 embedded hypervisor for the
		Intel x86 architecture.
	\item VxWorks MILS Platform by WindRiver Inc. is a type 1 hypervisor-based,
		SKPP-conformant MILS separation kernel.
	\item PolyXene by Bertin Technologies, is a microkernel-based hypervisor
		EAL-5 certified against the Guest Operating System Hosting Framework
		protection profile.
\end{itemize}

None of these companies publish detailed technical documentation or provide
access to source code. Thus there is not enough information available for a
thorough technical analysis to assess the assurance provided by these kernels
(e.g. if the kernels are suitable for formal verification and what kind of
verification has been performed). Simply put, there is not enough information
to verify the high robustness claims made by the vendors.
