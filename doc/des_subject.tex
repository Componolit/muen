\section{Subject}\label{sec:subject}
A subject is an untrusted software component that is executed by the separation
kernel. Similar terms used in literature are partition or component. Besides the
separation kernel itself, subjects constitute the majority of a system based on
the Muen kernel.

The main purpose of the kernel is to execute an arbitrary number of subjects,
giving them access to assigned resources and only allowing communication between
subjects via explicitly defined channels. The kernel manages subject execution
and treats all subjects equally.

Information related to a subject is categorized into two distinct categories:

\begin{description}
	\item[Specification] encompasses all static configuration data, that is
		constant and does not change during the runtime of the system, e.g.
		assigned hardware devices and memory resources.
	\item[State] is made up of all transient values that are potentially
		modified by the execution of a subject, e.g. CPU register values.
\end{description}

\subsection{Subject specification}
The specification defines, what resources a subject is allowed to access, what
specific execution environment the kernel must provide for the subject, the
initial state and of course the subject binary itself. All resources that a
subject can access, must be explicitly defined in the specification, which
includes the complete definition of the subject's memory layout.

This information is part of the overall system specification and is fixed at
integration time. It does not change during the execution of the system. The
kernel keeps this information as part of the compiled system policy in read-only
memory. Since subjects are unable to access kernel memory, the subject
specifications cannot be tampered with.

\subsection{Subject state}
Running a subject inevitable effects and changes the execution environment. The
state of a subject encompasses all system elements that are visible to the
subject and can contain different information depending on the current subject.
In particular this includes CPU registers, instruction and stack pointer as well
as control register values.

Since the kernel can pause and resume subject execution and potentially multiple
subjects can be executed in arbitrary order, the state of the system as viewed
by the subject must be saved. The state must be restored accurately upon
resumption otherwise the subject will not be able to execute seamlessly.
Additionally, if a subject is not oblivious to the fact that the kernel is
suspending its execution by saving and loading its state, the isolation of the
subject would be insufficient since this would be an unintended information
flow.

Subject management information used by the kernel during runtime is also
considered part of the subject state. This includes a data structure storing
pending interrupts to be delivered to the subject.

\subsection{Subject profile}
Two types of subjects are distinguished:

\begin{itemize}
	\item Native applications
	\item Virtual Machine (VM\index{VM}) subjects
\end{itemize}

Each of these types is captured by so called \emph{subject
profiles}\index{subject profiles}. These profiles determine the execution
environment and architectural features (e.g.  memory management) the subject is
allowed to use. Profiles are declared in the subject specification. The
currently implemented profiles are described in the following sections.

\subsubsection{Native subject}
A \emph{native} subject is a 64-bit application, that executes directly on the
processor without any supporting operating system kernel or runtime environment.
Such applications may also be called bare bones, bare metal or bare machine.

The execution environment of the native subject profile has the following main
properties:

\begin{itemize}
	\item IA-32e/64-bit processor mode
	\item No mode switching
	\item No memory management (static paging structures)
	\item No hardware exception handling
	\item No control register access
\end{itemize}

\subsubsection{VM subject}
A \emph{virtual machine} (VM) subject has more control over its execution
environment. The VM profile is intended for running operating systems such as
xv6 or Linux.

The execution environment of the VM subject profile has the following main
properties:

\begin{itemize}
	\item Switching between 32-bit and 64-bit modes
	\item Memory management and page table management via EPT
	\item Hardware exception handling
	\item Restricted control register access
\end{itemize}

The profile of a subject is part of the subject specification.
