\subsection{External interrupts}\label{subsec:external-ints}
Figure \ref{fig:external-interrupt} shows a schematic overview of the external
interrupt delivery mechanism provided by the Muen kernel. Interrupt requests
(IRQ\index{IRQ}) emitted by a hardware device are routed to a subject through
the kernels \texttt{Handle\_Irq} procedure.

\begin{figure}[h]
	\centering
	\begin{tikzpicture}
	\node[graybox] (han) {\textbf{3} Handle IRQ};
	\node[above=2mm of han] (mue) {Muen SK};

	\begin{pgfonlayer}{background}
		\node[bluebox, minimum width=3cm, minimum height=1.7cm] (mub) [fit = (han) (mue)] {};
	\end{pgfonlayer}

	\node[greenbox, minimum width=3cm, minimum height=1.7cm, above=of mub] (sub) {Subject};
	\node[apribox, left=15mm of mub] (irq) {Keyboard};

	\draw[arrow] (irq) to node[auto] {\textbf{1} IRQ 1} (mub);
	\draw[arrow] (sub.225) to node[auto, swap] {\textbf{2} VM exit} (mub.135);
	\draw[arrow] (mub.45) to node[auto, swap] {\textbf{4} Inject event} (sub.315);
\end{tikzpicture}

	\caption{External interrupt handling}
	\label{fig:external-interrupt}
\end{figure}

To make external interrupt routing work, the kernel programs the system's I/O
APIC on startup. Since each logical CPU of the system runs an identical kernel,
device IRQs must be forwarded to the LAPIC of the logical processor running the
subject in question.

Which device IRQ is routed to which subject is determined by the assignment of
hardware devices to subjects in the system policy (see section
\ref{subsec:subject_type}). The policy compilation tool uses this information to
compile the interrupt routing tables. These tables are SPARK specifications
which are compiled into the kernel during the build process.

The first table contains the IRQ routing information, i.e. to which CPU's LAPIC
the hardware IRQ must be forwarded to, and also the interrupt vector to use.
Hardware interrupts are remapped by the I/O APIC to the interrupt vector range
32-255 to be distinct from exceptions. For example, IRQ 1 used by the keyboard
is remapped to interrupt vector 33 before delivery to the assigned processor
LAPIC.

On receipt of the interrupt message, the LAPIC forwards the interrupt vector to
the processor core for further processing. It is important to note that
interrupts are not enabled when in VMX root mode (the interrupt flag
IF\index{IF} is not set in the host's FLAGS register). This simplifies the
kernel code and assures that the kernel is not disrupted by external interrupts.

Instead, the VT-x external interrupt exiting feature is used to pass control to
the kernel on interrupts. The activation of this feature leads to a VM exit
into the kernel with the VM exit reason set to "external interrupt" (1).  The
programming of the I/O APIC ensures that a CPU only receives interrupts
destined for a subject scheduled on this particular core.

If the VM exit reason indicates the occurrence of an external interrupt, the VM
exit handler of the kernel scheduler invokes the \texttt{Handle\_Irq} procedure.
The generated interrupt routing table is used to determine the destination
subject for the given interrupt vector. The interrupt is delivered to the
designated subject using the interrupt injection mechanism described in section
\ref{subsec:int-injection}.
